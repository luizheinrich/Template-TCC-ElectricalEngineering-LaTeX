\chapter{Introdução}      \label{Introducao}

Dapibus gravida tristique sodales purus condimentum porttitor, aliquam vulputate condimentum donec sapien justo praesent, sociosqu pellentesque dictum eros auctor. odio amet sem pretium eros facilisis curabitur velit tempus sapien, sodales praesent rutrum interdum tincidunt habitant euismod augue, tristique vehicula tempus molestie at quisque erat potenti. lacinia pulvinar class dictumst suspendisse eget etiam, molestie lectus class aenean purus eros primis, quam purus lectus viverra est. ante eget pretium lacus torquent cras ullamcorper neque, elit platea diam nulla potenti class auctor lectus, tempor dapibus a justo aptent rhoncus. praesent aliquet purus felis nostra pellentesque odio quisque praesent porttitor, curae maecenas placerat nostra maecenas erat ac tristique, iaculis porttitor habitant aptent suscipit posuere accumsan curabitur \cite{alexander2013fundamentos}.


\section{Trabalhos Relacionados}

Dapibus gravida tristique sodales purus condimentum porttitor, aliquam vulputate condimentum donec sapien justo praesent, sociosqu pellentesque dictum eros auctor. odio amet sem pretium eros facilisis curabitur velit tempus sapien, sodales praesent rutrum interdum tincidunt habitant euismod augue, tristique vehicula tempus molestie at quisque erat potenti. lacinia pulvinar class dictumst suspendisse eget etiam, molestie lectus class aenean purus eros primis, quam purus lectus viverra est. ante eget pretium lacus torquent cras ullamcorper neque, elit platea diam nulla potenti class auctor lectus, tempor dapibus a justo aptent rhoncus. praesent aliquet purus felis nostra pellentesque odio quisque praesent porttitor, curae maecenas placerat nostra maecenas erat ac tristique, iaculis porttitor habitant aptent suscipit posuere accumsan curabitur.


\section{Organização do Trabalho}

Este trabalho está organizado como se segue.

O Capítulo \ref{Revisao Bibliografica} é dividido em duas principais partes: na Seção \ref{Assunto 1} é feita uma revisão dos conceitos básicos da nanofotônica; e na Seção \ref{Assunto 2} são apresentados os conceitos básicos de redes neurais artificiais.

No Capítulo \ref{Metodo Proposto} é vista a metodologia de aplicação do método de modelagem inversa e otimização.

O Capítulo \ref{Resultados} mostra os resultados obtidos após a implementação do método de otimização proposto neste trabalho.

Por fim, no Capítulo \ref{Consideracoes Finais}, são mostradas as considerações finais desde trabalho.